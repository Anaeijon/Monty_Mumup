\newglossaryentry{raspberrypi}{
	name={Raspberry Pi},
	description={
		Eine sehr bekannte Reihe von \gls{open-source} \glspl{sbc} welche durch die Raspberry Pi Foundation entwickelt werden}
}

\newglossaryentry{sbc}{
	name={SBC},
	description={
		Ein Computer, bei welchem alle für den Betrieb benötigten Komponenten auf einer Platinen zusammengefasst sind, wird als \glsentrylong{sbc} (\glsentryname{sbc}, Einplatinencomputer) bezeichnet},
	plural={SBC},
    long={Single-board Computer},
    first={\glsentrylong{sbc}} (\glsentryname{sbc}),
    firstplural={Single-board Computern (\glsentryname{sbc})}
}

\newglossaryentry{iot}{
	name={IoT},
	description={
	Eingebettete Computer in alltäglichen Gegenständen können Daten sammeln und sich über das Internet vernetzen, um den Nutzer über die gewohnte Funktionalität hinaus zu unterstützen, dabei jedoch nicht aufzufallen },
    long={Internet of Things},
    first={\glsentrylong{iot} (\glsentryname{iot})},
}


\newglossaryentry{uuid}{
	name={UUID},
	description={
	Ein \glsentrylong{uuid} ist eine 128-Bit-Zahl die dazu dient, Computer und Daten weltweit eindeutig zu identifizieren. Durch den verwendeten Prozess bei der Erstellung einer \glsentryname{uuid} soll mit annähernd absoluter Sicherheit garantiert werden, dass es unter UUIDs keine Duplikate gibt},
    long={Universally Unique Identifier},
    first={\glsentrylong{uuid} (\glsentryname{uuid})},
}

\newglossaryentry{xor}{
	name={XOR},
	description={
	Eine XOR-Verknüpfung bezeichnet eine bitweise Operation bei der eine exklusive Disjunktion angewendet wird.
	Bitweise werden beide Eingaben verglichen. Sind die Eingaben an dieser Stelle gleich, so ist das Resultat an der Stelle 0. Unterscheiden sich die Eingaben, ist das Resultat an der Stelle 1.
	Beide Eingaben produzieren gemeinsam eine Ausgabe. Wendet man XOR auf eine der Eingaben und die vorherige Ausgabe an, so ist die andere Ausgabe das Resultat},
}

\newglossaryentry{hash}{
	name={Hash},
	description={
	Eine Hashfunktion bildet eine große Datenmenge auf eine kleinere ab. Es ist damit möglich, bis zu einem gewissen Grad an Sicherheit, eine Eingabe eindeutig zu identifizieren, ohne die Eingabe vollständig speichern oder übertragen zu müssen}
}

\newglossaryentry{dac}{
	name={DAC},
	description={
	Digital-Analog-Wandler (engl. Digital-Analog-Converter) sind Module, welche digitale zu analogen Signalen umsetzen. Sie werden hauptsächlich eingesetzt um in elektronischen Geräten analoge Signale für Lautsprecher zu erzeugen, welche als Schallwellen wiedergegeben werden},
	short={DAC},
    long={Digital-Analog-Wandler},
    first={\glsentrylong{dac}} (\glsentryname{dac}),
}


\newglossaryentry{i2s}{
	name={I$^2$S},
	description={\glsentryname{i2s} steht für \glsentrylong{i2s} und ist ein Standard für die digitale Übertragung von Tönen zwischen Prozessoren},
	long={Inter-IC Sound},
	short={I$^2$S},
	first={\glsentrylong{i2s}} (\glsentryname{i2s})
}

\newglossaryentry{distribution}{
	name={Distribution},
	description={
	Eine Distribution ist eine Bündelung von Software, welche einander benötigt, um sinnvoll genutzt werden zu können. Meist wird der Begriff benutzt um verschiedene Varianten von Linux-Systemen zu beschreiben, die neben dem Linux-Kernel noch weitere Softwarepakete enthalten um als vollständiges Betriebssystem nutzbar zu sein},
	plural={Distributionen}
}

\newglossaryentry{headless}{
	name={headless},
	description={
		Computer, welche über keine grafische Oberfläche verfügen werden als \enquote{\gls{headless}} bezeichnet}
}

\newglossaryentry{open-source}{
	name={Open-Source},
	description={
		Software deren Quellcodes oder Hardware deren technische Spezifikationen und Layouts von jedem eingesehen, verbreitet, modifiziert und genutzt werden können wird als \gls{open-source} bezeichnet }
}

\newglossaryentry{python}{
	name={Python},
	description={Python ist eine höhere Interpretersprache. Sie unterstützt funktionales, objektorientiertes und aspektorientiertes Programmieren  }
}


\newglossaryentry{api}{
    name={API},
    description={
    	Ein Application Programming Interface (API) ist eine Menge von Regeln und Spezifikationen, die ein Programm befolgen kann, um auf bestimmte Dienste und Ressourcen zugreifen zu können, die von einem anderen Programm bereit gestellt werden, welches die API implementiert},
    long={Application Programming Interface},
    first={\glsentrylong{api}} (\glsentryname{api}),
    firstplural={\glsentrylong{api}\glspluralsuffix\ (\glsentryname{api}\glspluralsuffix )}
}

\newglossaryentry{cli}{
    name={CLI},
    description={
    	Ein Command Line Interface (CLI) ist eine Schnittstelle zu zu einem Programm, welche über die Kommandozeile aufgerufen und genutzt werden kann},
    long={Command Line Interface},
    first={\glsentrylong{cli}} (\glsentryname{cli}),
    firstplural={\glsentrylong{cli}\glspluralsuffix\ (\glsentryname{cli}\glspluralsuffix )}
}

\newglossaryentry{html}{
    name={HTML},
    description={
    	Hypertext Markup Language (HTML) ist eine \gls{xml}-basierte Auszeichnungssprache. Sie die Grundlage für Seiten im \gls{www} und wird von \glspl{browser} interpretiert},
    long={Hypertext Markup Language},
    first={\glsentrylong{html}} (\glsentryname{html})
}


\newglossaryentry{js}{
    name={JavaScript},
    description={
  		JavaScript (JS) ist eine von \glspl{browser} interpretierte Programmiersprache, welche das Generieren von dynamischen \glspl{website} ermöglicht}
}


\newglossaryentry{json}{
    name={JSON},
    description={
    	\glsentrylong{json} (\glsentryname{json}) ist das Standarddatenformat von \gls{js} und zeichnet sich durch kompakte, klar lesbare und einfach interpretierbare notation von Objekten aus},
    long={JavaScript Object Notation},
    first={\glsentrylong{json} (\glsentryname{json})}
}


\newglossaryentry{dom}{
	name={DOM},
	description={
		Das \glsentrylong{dom} (\glsentryname{dom}) ist eine in \gls{js} enthaltene \gls{api} um den \gls{html}-Code der aufrufenden Seite in Form von Objekten zu laden und dynamisch zu verändern},
	long={Document Object Model},
	first={\glsentrylong{dom} (\glsentryname{dom})}
}

\newglossaryentry{ajax}{
	name={AJAX},
	description={
		\glsentrylong{ajax} (\glsentryname{ajax}) bezeichnet asynchrone Datenübertragungen zwischen \gls{js}-basierten Webanwendungen und einem Server
	},
	long={Asynchronous JavaScript and XML},
	first={\glsentrylong{ajax} (\glsentryname{ajax})}
}

\newglossaryentry{sse}{
	name={SSE},
	description={
		\glsentrylong{sse} (\glsentryname{sse}) bezeichnet eine in \gls{html}5 enthaltene Möglichkeit, über einen zuvor vom Client geöffneten Stream, Daten vom Server an den Client zu senden
	},
	long={Server-Send Events},
	first={\glsentrylong{sse} (\glsentryname{sse})},
	firstplural={\glsentrylong{sse}\glspluralsuffix\ (\glsentryname{sse}\glspluralsuffix )}
}

\newglossaryentry{websocket}{
	name={Websocket},
	description={
		Websockets sind eine Erweiterung des klassischen \gls{http}-Protokolls, die eine bidirektionale Kommunikation zwischen Client und Server ermöglichen},
	plural={Websockets}
}

\newglossaryentry{http}{
	name={HTTP},
	description={
		Das \glsentrylong{http} (\glsentryname{http}) ist das Übertragungsprotokoll, welches genutzt wird, um im \gls{www} \glspl{website} von einem Server zu laden},
	long={Hypertext Transfer Protocol},
	first={\glsentrylong{http} (\glsentryname{http})}
}


\newglossaryentry{css}{
    name={CSS},
    description={
    	\glsentrylong{css} (\glsentryname{css}) ist eine Sprache zur Definition von Styles für \glossary{html}-Elemente},
    long={Cascading Style Sheets},
    first={\glsentrylong{css} (\glsentryname{css})}
}


\newglossaryentry{xml}{
    name={XML},
    description={
    	\glsentrylong{xml} (engl. Erweiterbare Auszeichnungssprache)},
    long={Extensible Markup Language},
    first={\glsentrylong{xml} (\glsentryname{xml})}
}


\newglossaryentry{www}{
    name={WWW},
    description={
    	Das World Wide Web (WWW) besteht aus \glspl{website}, die per Hyperlinks miteinander verknüpft sind und über das Internet abgerufen werden können},
    long={World Wide Web},
    first={\glsentrylong{www} (\glsentryname{www})}
}


\newglossaryentry{website}{
	name={Webseite},
	description={
		Über \glspl{url} erreichbares Dokument, das mit \gls{html} formatiert wurde und von \glspl{browser} aufgerufen werden kann},
	plural={Webseiten}
}

\newglossaryentry{webapp}{
	name={Web-App},
	description={
		In \gls{html} und \gls{js} umgesetzte Anwendungen, welche ohne Installation von einem \gls{browser} geladen und ausgeführt werden kann},
	plural={Web-Apps}
}

\newglossaryentry{browser}{
	name={Webbrowser},
	description={
		Programm zur Darstellung von \glspl{website}},
	plural={Webbrowser}
}


\newglossaryentry{url}{
	name={URL},
	description={
		\glsentrylong{url}\glspluralsuffix\ (\glsentryname{url}\glspluralsuffix) (engl. einheitlicher Ressourcenzeiger) dienen im \gls{www} der eindeutigen Definition und referenzierung von Ressourcen},
	plural={URLs},
	long={Uniform Resource Locator},
	first={\glsentrylong{url}} (\glsentryname{url}),
	firstplural={\glsentrylong{url}\glspluralsuffix\ (\glsentryname{url}\glspluralsuffix )}
}

\newglossaryentry{koroutine}{
	name={Koroutine},
	description={Prozeduren, deren Ablauf unterbrochen und bei Bedarf wieder aufgenommen werden kann},
	plural={Koroutinen}
}
